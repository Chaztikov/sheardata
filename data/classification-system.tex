% Copyright (C) 2020 Andrew Trettel
%
% This file is licensed under the Creative Commons Attribution 4.0
% International license (`CC-BY-4.0`).  For more information, please visit the
% Creative Commons website at <https://creativecommons.org/licenses/by/4.0/>.

\section{Classes}

\input{text-flow-classes.tex.tmp}

\begin{figure}
    \centering
    \input{figure-tree-diagram.tex.tmp}
    \caption{Tree diagram of flow classification system using codes for the
    flow classes}
\end{figure}

\section{Principles}

Any practical taxonomy should organize information in a useful manner.  This
project tries to organize a wide variety of data from different shear layers
into a unified framework, so it requires some means to distinguish between
different kinds of flows.  Previous classification systems have tended to be
too complex, with too many classes and too much ambiguity.  A simpler
classification scheme with a limited number of classes will be much easier to
use.  To achieve this goal, this taxonomy follows 5 principles (language
following RFC 2119):

\begin{enumerate}


\item
Each flow class MUST be identified by a single letter related to the flow's
name.

This principle limits the maximum number of classes to 26\footnote{
%%%%%%%%%%%%%%%%%%%%%%%%%%%%%%%%%%%%%%%%
Strictly speaking, only 25 classes are available.  To reduce ambiguity, the
classification MUST not use letter \texttt{O}, since it is easily confused with the
number \texttt{0}.
%%%%%%%%%%%%%%%%%%%%%%%%%%%%%%%%%%%%%%%%
} and creates a simple and obvious code to identify a study's class.  Moreover,
using letters for the code allows changes to the classification much more
readily than using hierarchical information for the code (the location of the
class in the classification).  The choice of letters means that if the
classification needs to be changed, it can be changed without having to update
the codes.

A brief explanation of unexpected codes:

    \begin{itemize}

    \item
    \texttt{C} for wall-bounded flows from ``confinement''.

    \item
    \texttt{G} for isotropic flows from ``grid turbulence''.

    \item
    \texttt{K} for wall jets since it is the next letter after \texttt{J}, the
    code for free jets.

    \item
    \texttt{N} for inhomogeneous flows from its second letter.

    \item
    \texttt{U} for flows from ``unclassified'' flows.

    \end{itemize}


\item
Individual studies MUST belong to a single class.

For example, suppose that there are two classes for boundary layers: attached
boundary layers and separated boundary layers.  It is possible that a single
study contains a series of boundary layer profiles that separate and then
reattach.  Should the entire series of profiles be classified as separated even
though only some are?  Should each profile be classified individually instead?
No, the simplest solution is to create only one class for boundary layers.
This choice makes classifying a given study much easier.


\item
Each flow class MUST work for all coordinate systems.

This principle prevents additional classes being created for each coordinate
system.  For example, both round jets and plane jets are jets first and
foremost, so there is no need for an additional class for each.  Similarly,
both pipe flows and rectangular duct flows are classified as duct flows since
they are duct flows first and foremost.  Additional fields should specify the
coordinate system and geometry of the boundaries, to keep track of this
information outside of the classification itself.


\item
The differences between classes SHOULD be intrinsically discrete and
unquantifiable.

For example, there should be a single class for boundary layers, and not
multiple classes (incompressible laminar boundary layers, incompressible
turbulent boundary layers, compressible laminar boundary layers, and
compressible turbulent boundary layers).  In each case, the difference between
those classes can be largely quantified with the Reynolds and Mach numbers, so
there is no need to create additional classes when the effect can be quantified
continuously using an additional variable rather than discretely with a
different class.

By this principle, a plume (buoyant jet) is classified under class \texttt{J}
(free jets).  Buoyancy is quantifiable and therefore a plume, as a type of jet,
does not receive a separate class.

Note that internal flows violate this principle since the amount of relative
motion can in fact be quantified, but I split internal flows into two classes
(\texttt{D} and \texttt{R}) anyway.  In this case, most internal flow studies
do not have both a pressure gradient and relative motion of the boundaries, so
it is more practical to divide internal flows this way.


\item
The classification system SHOULD emphasize differences in flow pattern and
geometry over differences in flow mechanism and physics.

This principle guides the large-scale structure of the classification more than
small-scale structure.  The point is that the classification divides flows up
more by their flow pattern (homogeneity, kinematics, and presence of
boundaries) than by the specific physics driving the flow (the flow mechanism)
or especially the types of additional physics in the flow (shock waves,
separation, cavitation, free-stream turbulence, \ldots).

The overall structure of the tree diagram emphasizes this principle by
immediately considering homogeneity (geometry), then shear (kinematics), and
then the influence of boundaries (geometry again).  Indeed, no class is defined
exclusively by the presence of additional physics.


\end{enumerate}
